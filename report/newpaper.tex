
%%%%%%\documentclass[times, 10pt,twocolumn, english]{article} 
\documentclass[10pt, conference, compsocconf, english]{IEEEtran}

% Add the compsocconf option for Computer Society conferences.
%
% If IEEEtran.cls has not been installed into the LaTeX system files,
% manually specify the path to it like:
% \documentclass[conference]{../sty/IEEEtran}
%%%%%%%%\usepackage{latex8}
\usepackage{times}
\usepackage[tight,footnotesize]{subfigure}
\usepackage{bibspacing}
\setlength{\bibspacing}{\baselineskip}

%%%%%%%%%%%%%%%%%%%%%%%
\usepackage{verbatim}
\usepackage{amsmath}
\usepackage{babel}
\usepackage{listings}
\usepackage[]{graphicx} 
\usepackage{comment} 

%% Math Packages %%%%%%%%%%%%%%%%%%%%%%%%%%%%%%%%%%%%%%%%%%%%
\usepackage{amsmath}
\usepackage{amsthm}
\usepackage{amsfonts}

\newcommand{\mySection}[1]{\vspace{-0pt}\section{\hskip -1em.~~#1}\vspace{-00pt}}
\newcommand{\mySubSection}[1]{\vspace{-0pt}\subsection{\hskip -1em.~~#1}\vspace{-00pt}}
\newcommand{\mySubSubSection}[1]{\vspace{-0pt}\subsubsection{\hskip -1em.~~#1}\vspace{-0pt}}
\newcommand{\tableref}[1]{Table~\ref{tab:#1}}
\newcommand{\figref}[1]{Fig.~\ref{fig:#1}}

% correct bad hyphenation here
\hyphenation{op-tical net-works semi-conduc-tor}

%\setlength{\textheight}{9.2in}
%\setlength{\textwidth}{7.3in}
%\setlength{\columnsep}{0.15in}
%\setlength{\topmargin}{-0.0in}
%\setlength{\headheight}{-0.1in}
%\setlength{\headsep}{-0in}
%\setlength{\parindent}{1pc}
%\setlength{\oddsidemargin}{-.204in}
%\setlength{\evensidemargin}{-.204in}

\pagestyle{empty}

\begin{document}
%
% paper title
% can use linebreaks \\ within to get better formatting as desired
\title{New Paper Title Here}


% author names and affiliations
% use a multiple column layout for up to two different
% affiliations

\author{Author1 and Author2\\
Department of Electrical and System Engineering\\ University of Pennsylvania \\ \{author1, author2\}@seas.upenn.edu\\
}

% use for special paper notices
%%%%\IEEEspecialpapernotice{(Invited Paper)}

% make the title area
\maketitle
\thispagestyle{empty}


\begin{abstract}
%%%%%% \vspace{-10pt}
Abstract goes here....
\end{abstract}


  \vspace{-10pt}
\mySection{Introduction}
\label{sec:intro_sec}
  \vspace{-5pt}
Intro goes here...

\mySubSection{My SubSection}
My Sub Section goes here...

%%%%%%%%%%%%%%%%%%%%%%%%%%%%%%%%%%%%%%%%%%%%%%%%%%%%%%
\begin{figure}[!b]
		\centering
		\vspace{-15pt}
		\includegraphics[width=0.49\textwidth]{figs/taskmigrate.pdf}
		\vspace{-20pt}
		\caption{\small Task migration for real-time operation  (instructions, stack, data \& timing/fault tolerance meta-data) on one physical node to another. }
		  \vspace{-10pt}
		\label{fig:taskmigrate}
\end{figure} 
%%%%%%%%%%%%%%%%%%%%%%%%%%%%%%%%%%%%%%%%%%%%%%%%%%%%%%


%%%%%%%%%%%%%%%%%%%%%%%%%%%%%%%%%%%%%%%%%%%%%%%%%%%%%%
%\begin{figure*}[!t]
%	\begin{center}
%	\subfigure [EVM Design flow] {
%				\includegraphics[width=0.25\textwidth]{figs/design_flow.pdf} 
%				\label{fig:des_flow}
%		}
%		\subfigure [A control algorithm (VT) modeled in Simulink] {
%				\includegraphics[width=0.35\textwidth]{figs/SimulinkVT.pdf} 
%				\label{fig:SimVT}
%		}
%		\subfigure [Generated code] {	
%			\includegraphics[width=0.3\textwidth]{figs/SimulinkVTcode.pdf} 
%			\label{fig:SimVTcode}
%		}
%	\end{center}
%	\label{fig:SimulinkVT}
%	\vspace{-10pt}
%	\caption{\small Generation of EVM functional description from Simulink model}
%	\vspace{-15pt}
%\end{figure*} 
%%%%%%%%%%%%%%%%%%%%%%%%%%%%%%%%%%%%%%%%%%%%%%%%%%%%%%

\mySubSection{Another Sub Section}
\vspace{-5pt}
Another sub section goes here

\mySection{Related work}
\label{sec:related_sec}
Related work goes here

\mySection{Conclusion}
\label{sec:conc_sec}
\vspace{-5pt}
Conclusion goes here

\vspace{-5pt}
\bibliographystyle{unsrt}
{ \small 
\bibliography{bibliography}
}
\end{document} 


% that's all folks
\end{document}

%%%%%
%%%%%notes:


